\documentclass[a4paper,11pt]{scrartcl}
\usepackage[dvips]{graphicx}
\usepackage[twoside,paper=a4paper,hmarginratio=3:2,tmargin=2.5cm,bmargin=3cm]{geometry}
\usepackage{scrpage2}
\usepackage{amsmath,amsbsy,amsfonts,amssymb,amsxtra}
\usepackage{enumitem}
\usepackage{setspace}
\usepackage{gb_custom}

\setlength\marginparsep{0cm}

\graphicspath{{./figures/}}


\reversemarginpar


\author{C.S.~Allen, M.~Newman and G.~Burnell}
\title{Stoner Python Module}

\begin{document}

\maketitle

\tableofcontents
\newpage
\pagestyle{scrheadings} \ihead[Stoner Python Module]{Stoner Python Module} \ifoot[\today]{\today}
\ohead[Manual]{Manual}



  \section{Introduction}

This manual provides a user guide and reference for the Stoner python module. The Stoner python module provides a set of python classes and functions for reading, manipulating and plotting data acquired with the lab equipment in the Condensed Matter Physics Group at the University of Leeds.

\subsection{Getting the Stoner module}

The source code for the Stoner python module is kept in CVS revision control on the stonerlab server. A stable release of the code is available for copying and use in \verb#\\stonerlab\data\software\python\stable\#. The development code can be obtained by checking out the PythonCode module with a CVSROOT of \\ \verb#:ext:cvs@stonerlab.leeds.ac.uk:/home/cvs/#. Appropriate ssh keys for the cvs user account are kept in \verb#\\stonerlab\data\software\CVS\#.

\subsection{Using the Stoner module}

The easiest way to use the Stoner Module is to add the path to the directory containing Stoner.py to your PYTHONPATH environment variable. This can be done on Macs and Linux by doing:
\begin{verbatim}
  cd <path to PythonCode directory>/src
  export PYTHONPATH=`pwd`:$PYTHONPATH
\end{verbatim}
On a windows machine the easiest way is to create a permanent entry to the folder in the system environment variables. Go to Control Panel -> System -> Advanced Tab -> click on Environment button and then add or edit an entry to the system variable PYTHONPATH.

One this has been done, the Stoner module may be loaded from python command line:

\begin{verbatim}
  >>> import Stoner
\end{verbatim}

or

\begin{verbatim}
  >>> from Stoner import *
\end{verbatim}

The Stoner module currently depends on a number of other modules. These are installed on the lab machines that have Python installed. Primarily these are Numpy, SciPy and Matplotlib. Windows installable versions are kept in \\ \verb#\\stonerlab\\data\software\Python for Windows\#.

\section{Users' Guide}

The Users'Guide provides a brief overview of the functions contained within the Stoner module and so basic examples of how the module can be used.

The Stoner module provides several Python classes that can be used to manipulate experimental data. The main class that provides the basic functionality is the DataFile class. This handles loading data, finding and manipulating meta data, selecting rows or columns of data, adding or removing data, and saving data.

The PlotFile class is a descendent of DataFile, meaning it shares all the same functionality as DataFile, but in addition has methods to present data graphically. The AnalysisFile class is another descendent of DataFile, but provides extra methods to fit curves and carry out other simple analysis operations. Finally the FilterFile class provides extra methods to manipulate whole columns of data.

\subsection{Loading a data file}

The first step in using the Stoner module is to load some data from a measurement.

\begin{verbatim}
  >>>import Stoner
  >>>d=Stoner.DataFile('my_data.txt')
\end{verbatim}

In this example we have loaded data from my\_data.txt which should be in the current directory -- here we are assuming that my\_data.txt contains data in the \textit{TDI Format 1.5} which is produced by the LabVIEW rigs. Assuming that the file successfully loads, \textit{d}, is an instance of the DataFile object. Here the DataFile constructor has been used to both create the instance and load the data in one go. It is also possible to do these steps separately, or indeed to load new data into an existing instance of DataFile.

\begin{verbatim}
  >>>import Stoner
  >>>d=Stoner.DataFile()
  >>>d.load('my_data.txt')
\end{verbatim}

\subsection{Examining Some Data}

Having loaded some data, the next stage might be to take a look at it. Internally, data is represented as a 2D numpy array of floating point numbers, along with a list of column headers and two dictionaries that hold metadata and type information about metadata (metametadata perhaps !). These can be accessed like so:
\begin{verbatim}
  >>>d.data
  >>>d.column_headers
  >>>d.metadata
  >>>d.typehint
\end{verbatim}
This is all very well, but often you want to examine a particular column of data or a particular row:
\begin{verbatim}
  >>>d.column(0)
  >>>d.column('Temperature')
  >>>d.column(['Temperature',0])
\end{verbatim}
In the first example, the first column of numeric data will be returned. In the second example, the column headers will first be checked for one labeled exactly \textit{Temperature} and then if no column is found, the column headers will be searched using \textit{Temperature} as a regular expression. This would then match \textit{Temperature (K)} or \textit{Sample Temperature}.  The third example results in a 2 dimensional numpy array containing two columns in the order that they appear in the list (\ie not the order that they are in the data file).

Rows don't have labels, so are accessed directly by number:
\begin{verbatim}
  >>>d[1]
\end{verbatim}
What happens if you use a string and not a number in the above examples ?
\begin{verbatim}
  >>>d['User']
\end{verbatim}
in this case, it is assumed that you meant the metadata with key \textit{User}. To get a list of possible keys in the metadata, you can do:
\begin{verbatim}
  >>>d.dir()
  >>>d.dir('Option\:.*')
\end{verbatim}
In the first case, all of the keys will be returned in a list. In the second, only keys matching the pattern will be returned -- all keys containing \textit{Option:}.

Sometimes you may want to iterate over all of the rows or columns in a data set. This can be done quite easily:
\begin{verbatim}
  >>>for row in d.rows():
  ......print row
  ......
  >>>for column in d.columns():
  ......print column
  ......
\end{verbatim}
The first example could also have been written more compactly as:
\begin{verbatim}
  >>>for row in d:
  ......print row
  ......
\end{verbatim}

\end{document}