\documentclass[a4paper,11pt]{scrartcl}
\usepackage[dvips]{graphicx}
\usepackage[twoside,paper=a4paper,hmarginratio=3:2,tmargin=2.5cm,bmargin=3cm]{
geometry}
\usepackage{scrpage2}
\usepackage{amsmath,amsbsy,amsfonts,amssymb,amsxtra}
\usepackage{enumitem}
\usepackage{setspace}
\usepackage{hyperref}
\usepackage{MnSymbol}
\usepackage{gb_custom}


\lstset{frame=tblr,
	framerule=0mm,
	framesep=0.2mm,
	resetmargins=true,
	prebreak=\raisebox{0ex}[0ex][0ex]{\ensuremath{\rhookswarrow}},
	backgroundcolor=\color{white}
	linewidth=16cm,
	breaklines=true
	}

\setlength\marginparsep{0cm}

\graphicspath{{./figures/}}
\hypersetup{linkcolor=blue, colorlinks}%[colorlinks=true, linkcolor=blue]

\reversemarginpar


\author{C.S.~Allen, M.~Newman, R.~Temple, S.~Morely  and G.~Burnell}
\title{Stoner Python Package}

\begin{document}

\maketitle

\tableofcontents
\newpage
\pagestyle{scrheadings} \ihead[Stoner Python Package]{Stoner Python Package}
\ifoot[\today]{\today}
\ohead[Manual]{Manual}



  \section{Introduction}

This manual provides a user guide and reference for the Stoner python pacakage.
The Stoner python package provides a set of python classes and functions for
reading, manipulating and plotting data acquired with the lab equipment in the
Condensed Matter Physics Group at the University of Leeds.

\subsection{Getting the Stoner Package}

The easiest way to get and install the package is to make use of the EGG
package on PyPi. This will install a reasonably stable release into your
Python setup. Open a command prompt and run:

\begin{verbatim}
easy_install Stoner
\end{verbatim}

The advantage of getting the package this way is that it is installed into your Python path properly.
The disadvantage is that you don't get this user guide and the version may not be the most
up to date (although given the fragile and continuously being broken state of the code that may be
a good thing !).

\subsubsection{Getting the Latest Development Code}

\keypoint{These isntrctions are for members of the University of Leeds Condensed Matter Physics Group. External users are recommended to
download the source from GitHub}

The source code for the Stoner python module is kept on github using the git
revision control tool. A nightly development release of the code is available for copying and
use in \verb#\\stonerlab\data\software\python\PythonCode\#.

The Stoner Package currently depends on a number of other modules. These are installed on the lab
machines that have Python installed. Primarily these are Numpy, SciPy and Matplotlib.  The easiest way to get a Python
installation with all the necessary dependencies for the Stoner Package is to install the \textit{Enthought Python Distribution, Canopy}. Installers for Windows, MacOS and Linux are
kept in \\ \verb#\\stonerlab\data\software\Python#


\subsection{Using the Stoner Package}

\keypoint{If you have installed the Stoner Package with the easy\_install command
given above, then you can disregard this section.}

The easiest way to use the Stoner Package is to add the path to the directory
containing Stoner.py to your PYTHONPATH environment variable. This can be done
on Macs and Linux by doing:
\begin{lstlisting}
  cd <path to PythonCode directory>
  export PYTHONPATH=`pwd`:$PYTHONPATH
\end{lstlisting}
On a windows machine the easiest way is to create a permanent entry to the
folder in the system environment variables. Go to Control Panel -> System ->
Advanced Tab -> click on Environment button and then add or edit an entry to the
system variable PYTHONPATH.

One this has been done, the Stoner module may be loaded from python command
line:

\begin{lstlisting}
   import Stoner
\end{lstlisting}

or

\begin{lstlisting}
   from Stoner import *
\end{lstlisting}

\subsection{Documentation}

This document provides a user guide to the Stoner package and its various modules and classes. It is not a reference to the library but instead aims to explain the various operations that are possible and provide short examples of use. For the API reference for the library, please see the \textit{Python Code API} compiled windows help file. Rowan has also made a single sided ``cheat sheet'' that summarises the examples in this user guide.

\warning{The code is still under active development to fix bugs and add features. Generally things don't get deliberately broken, but accidents happen, so if something stops working, please either fix and commit the code or tell Gavin.}

\subsection{Users' Guide}

The Users'Guide provides a brief overview of the functions contained within the
Stoner module and so basic examples of how the module can be used.

The Stoner module provides several Python classes that can be used to manipulate
experimental data. The main class that provides the basic functionality is the
DataFile class. This handles loading data, finding and manipulating meta data,
selecting rows or columns of data, adding or removing data, and saving data.

The PlotFile class is a descendent of DataFile, meaning it shares all the same
functionality as DataFile, but in addition has methods to present data
graphically. The AnalyseFile class is another descendent of DataFile, but
provides extra methods to fit curves, smooth and differentiate data, find peaks
and carry out other simple analysis operations.

\section{Loading a data file}

The first step in using the Stoner module is to load some data from a
measurement.

\begin{lstlisting}
  import Stoner
  d=Stoner.DataFile('my_data.txt')
  d=Stoner.VSMFile('my_VSM_data.fld')
\end{lstlisting}

In this example we have loaded data from my\_data.txt which should be in the
current directory -- here we are assuming that my\_data.txt contains data in the
\textit{TDI Format 1.5} which is produced by the LabVIEW rigs. Assuming that the
file successfully loads, \textit{d}, is an instance of the DataFile object. Here
the DataFile constructor has been used to both create the instance and load the
data in one go.

The second example shows the use of one of the sub-classes of the DataFile object to load data from a specific instrument (in this case the VSM).

\warning{This is an API change from earlier versions of the Stoner package where a second parameter on the constructor of the DataFile object was used to identify the type of data file. This syntax is now depreciated !).}

The possible sub-classes are:
\begin{description}
\item[DataFile] Tagged Data Interchange Format 1.5 -- the default format produced by
the LabVIEW measurement rigs
\item[VSMFile] The text files produced by the group's Oxford Instruments VSM
\item[BigBlueFile] Datafiles produced by VB Code running on Big Blue. The
\textit{BigBlue} version of the DataFile.load and DataFile constructors takes
two additional parameters that specify the row on which the column headers will
be found and the row on which the data starts.
\item[CSVFile] Reads a generic comma separated value file. The \textbf{CSVFile} load
routine takes four additional parameters to the constructor and load methods. In
addition to the two extra arguments used for the \textit{BigBlue} variant, a
further two parameters specify the deliminators for the data and header rows. \textbf{CSVFile} also offers a \textbf{save} method to allow data to be saved in a simple deliminated text way (see Section \ref{save} for details).
\item[XRDFile] Loads a scan file produced by Arkengarthdale - the group's Brucker
XRD Machine.
\item[SPCFile] Loads a Raman scan file (.spc format) produced by the Rensihaw and Horiba
Raman spectrometers. This may also work for other instruments that produce spc files, but has not been extensively tested.
\item[BNLFile] Loads a SPEC file from Brookhaven (so far only tested on u4b files but may well work with other synchrotron data). Produces metadata Snumber: Scan number, Stype: Type of scan, Sdatetime: date time stamp for the measurement, Smotor: z motor position.
\item[TDMSFile] Loads a file saved in the National Instruments TDMS format
\item [QDSquidVSMFile] Loads data from a Quantum Design SQUID VSM as used on the I10 Beamline in Diamond.
\item [OpenGDAFile] Reads a scan file generated by OpenGDA -- a software suite used for synchtrons such as Diamond.
\item [RasorFile] Simply an alias for OpenGDAFile used for the RASOR instrument on I10 at Diamond.
\item [FmokeFile] Loads a file from Dan Allwood's Focussed MOKE System in Sheffield.
\end{description}

\begin{lstlisting}
  import Stoner
  d=Stoner.DataFile()
  d.load('my_data.txt')
   v=Stoner.VSMFIle()
  v.load('my_VSM_data.fld')
   c=Stoner.CSVFile()
   c.load('data.csv',1,0,',',',')
\end{lstlisting}

\keypoint{The load method, like many of the DataFile methods returns a copy of
the Datafile object \textbf{as well as} modifying the object itself. The
advantage of this is that it is then possible to chain several methods into one
command}

Sometimes you won't know exactly which subclass of \textbf{DataFile} is the one to use. Unfortunately, there is no sure fire way of telling, but \textbf{DataFile.load} will try to do the best it can and will try all of the subclasses in memory in turn to see if one will load the file without throwing an error. If this succeeds then the actual type of file that worked is stored in the metadata of the loaded file.

\warning{The automatic loading assumes that each load routine does sufficient sanity checking that it will throw and error if it gets bad data. Whilst one might wish this was always true it relies on whoever writes the load method to make sure of this ! If you want to stop the automatic guessing from happening use the auto\_load=False keyword in the \textbf{load} method.}

\subsection{Loading Data from a stromg or iterable object}

In some circumstances you may have a string representation of a \textbf{DataFile} object and want to transform this into a proper \textbf{DataFile} object.
This might be, for example, from transmitting the data over a network connection or receiving it from another program. In these situations the \textit{left shift operator}, \verb#<<#, can be used.

\begin{lstlisting}
   data=Stoner.Core.DataFile() << string_of_data
   data=Stoner.Core.DataFile() << iterable_object
\end{lstlisting}

The second example would allow any object that can be iterated (i.e. has a \textit{next()} method that returns lines of the data file, to be used
as the source of the data. The \textbf{Stoner.Core.DataFile()} creates an empty object so that the left shift operator calls the method
in \textbf{DataFile} to read the data in. It also determines the type of the object \verb#data#. This also provides an alternative syntax for reading a file
from disk:

\begin{lstlisting}
   data=Stoner.Core.DataFile()<<open("File on Disk.txt")
\end{lstlisting}

\section{Examining and Basic Manipulations of Data}
\subsection{Data Structure}
\subsubsection{Data, Column headers and metadata}
Having loaded some data, the next stage might be to take a look at it.
Internally, data is represented as a 2D numpy masked array of floating point numbers,
along with a list of column headers and a dictionary that keeps the metadata and
also keeps track of the expected type of the metadata (\ie the meta-metadata).
These can be accessed like so:
\begin{lstlisting}
  d.data
  d.column_headers
  d.metadata
\end{lstlisting}

\subsubsection{Masked Data and Why You Care}\label{(maskeddata)}
Masked data arrays differ from normal data arrays in that they include an option to mask or hide individual data elements. This can be useful to temporarily discount parts of your data when, for example, fitting a curve or calculating a mean value or plotting some data. One could, of course, simply ignore the masking option and use the data as is, however, masking does have a number of practical uses.

The data mask can be accessed via the \textit{mask} attribute of \textbf{DataFile}:
\begin{lstlisting}
   import numpy.ma as ma
   print d.mask
   d.mask=False
   d.mask=ma.nomask
   d.mask=numpy.array([[True, True, Fale,...False],...,[False,True,...True]])
   d.mask=lambda x: x[0]<50
   d.mask=lambda x:[y<50 for y in x]
\end{lstlisting}

The first line is simply the import statement for the numpy masked arrays in order to get the \textit{nomask} symbol. The second line will simply print the current mask. The next two examples will unmask all the data \ie make the values visible and useable. The next example illustrates using a numpy array of booleans to set the mask - every element in the mask array that evaluates as a boolean True will be masked and every False value unmasked. So far the semantics here are the same as if one had accessed the mask directly on the data via \verb'd.data.mask' but the final two examples illustrate an extension that setting the \textbf{DataFile} mask attribute allows. If you pass a callable object to the mask attribute it will be executed, passing each row of the data array to the user supplied function as a numpy array. The user supplied function can then either return a single boolean value -- in which case it will be used to mask the entire row -- or a list of booleans to mask individual cells in the current row.

By default when the \textbf{DataFile} object is printed or saved, data values that have been masked are replaced with a ``fill'' value of $10^{20}$.

\warning{This is somewhat dangerous behaviour. Be very careful to remove a mask before saving data if there is any chance that you will need the masked data values again later !}

\subsubsection{Working with columns of data}

This is all very well, but often you want to examine a particular column of data
or a particular row:
\begin{lstlisting}
  d.column(0)
  d.column('Temperature')
  d.column(['Temperature',0])
\end{lstlisting}
In the first example, the first column of numeric data will be returned. In the
second example, the column headers will first be checked for one labeled exactly
\textit{Temperature} and then if no column is found, the column headers will be
searched using \textit{Temperature} as a regular expression. This would then
match \textit{Temperature (K)} or \textit{Sample Temperature}.  The third
example results in a 2 dimensional numpy array containing two columns in the
order that they appear in the list (\ie not the order that they are in the data
file). For completeness, the \textbf{DataFile.column} method also allows one to
pass slices to select columns and should do the expected thing.

There is a convenient shortcut for working with cases where the column headers are not the same
as the names of any of the attributes of the \textbf{DataFile} object:
\begin{lstlisting}
  d.Temperature
  d.column('Temperature')
\end{lstlisting}
both return the same data.

Whenever the Stoner package needs to refer to a column of data, you cn specify it in a number of ways:
\begin{itemize}
\item As an integer where the first column on the left is index 0
\item As a string. if the string matches a column header exactly then the index of that column is returned. If the string fails
to match any column header it is compiled as a regular expression and then that is tried as a match. If multiple columns match then only
the first is returned.
\item As a regular expression directly - this is similar to the case above with a string, but allows you to pass a pre-compiled regular
expression in directly with any extra options (like case insensitivity flags) set.
\item As a slice object (ee.g. \verb#0:10:2#) this is expanded to a list of integers.
\item As a list of any of the above, in which case the column finding routine is called recursively in turn for each element of the list and
the final result would be to use a list of column indices.
\end{itemize}
\begin{lstlisting}
 import re
 col=re.compile('^temp',re.IGNORECASE)
 d.column(col)
\end{lstlisting}

\subsubsection{Working with complete rows of data}

Rows don't have labels, so are accessed directly by number:
\begin{lstlisting}
  d[1]
  d[1:4]
\end{lstlisting}
The second example uses a slice to pull out more than one row. This syntax also
supports the full slice syntax which allows one to, for example, decimate the
rows, or directly pull out the last fews rows in the file.

\subsubsection{Manipulating the metadata}

What happens if you use a string and not a number in the above examples ?
\begin{lstlisting}
  d['User']
\end{lstlisting}
in this case, it is assumed that you meant the metadata with key \textit{User}.
To get a list of possible keys in the metadata, you can do:
\begin{lstlisting}
  d.dir()
  d.dir('Option\:.*')
\end{lstlisting}
In the first case, all of the keys will be returned in a list. In the second,
only keys matching the pattern will be returned -- all keys containing
\textit{Option:}. For compatibility with normal opython semantics: \verb#d.keys()# is
synonymous with \verb#d.dir()#.

We mentioned above that the metadata also keeps a note of the expected type of
the data. You can get at the metadata type for a particular key like this:
\begin{lstlisting}
  d.metadata.type('User')
\end{lstlisting}
to get a dictionary of all of the types associated with each key you could do:
\begin{lstlisting}
  dict(zip(d.dir(),d.metadata.type(d.dir())))
\end{lstlisting}
but an easier way would be to use the \textbf{typeHintedDict.types} attribute:
\begin{lstlisting}
   d.metadata.types
\end{lstlisting}
.

\subsubsection{More on Indexing the data}

There are a number o other forms of indexing supported for \textbf{DataFile}
objects.

\begin{lstlisting}
  d[10,0]
  d[0:10,0]
  d[10,'Temp']
  d[0:10,['Voltage','Temp']
\end{lstlisting}

The first variant just returns the data in the 11th row, first column (remember
indexing starts at 0). The second variant returns the first 10 values in the
first column. The third variant demonstrates that columns can be indexed by
string as well as number, and the last variant demonstrates indexing
multiplerows and columns -- in this case the first 10 values of the Voltage and
Temp columns.

You might think of the data as being a list of records, where each column is a
field in the record. Numpy supports this type of structured record view of data
and the \textbf{DataFile} object provides the \textit{DataFile.records}
attribute to d this. This read-only attribute is just providing an alternative
view of the same data.

\begin{lstlisting}
   d.records
\end{lstlisting}

\subsubsection{Selecting Individual rows and columns of data}

Many of the function in the Stoner module index columns by searching the column
headings. If one wishes to find the numeric index of a column then the
\textbf{DataFile.find\_col} method can be used:

\begin{lstlisting}
   index=d.find_col(1)
   index=d.find_col('Temperature')
   index=d.find_col('Temp.*')
   index=d.find_col('1')
   index=d.find_col(1:10:2)
   index=d.find_col(['Temperature',2,'Resistance'])
\end{lstlisting}

 \textbf{DataFile.find\_col} takes a number of different forms. If the argument
is an integer then it returns (trivially) the same integer, a string argument is
first checked to see if it exactly matches one of the column headers in which
case the number of the matching column heading is returned. If no exact match is
found then a regular expression search is carried out on the column headings. In
both cases, only the first match is returned. If the string still doesn't match, then
the string is checked to see if it can be cast to an integer, in which case the integer value is used.

The final two examples given above
both return a list of indices, firstly using a slice construct - in this case
the result is trivially the same as the slice itself, and in the last example by
passing a list of column headers to look for.

This is the function that is used internally by \textbf{DataFile.column},
\textbf{DataFile.search} \etc and for this reason the trivial integer and slice
forms are implemented to allow these other functions to work with multiple
columns.

Sometimes you may want to iterate over all of the rows or columns in a data set.
This can be done quite easily:
\begin{lstlisting}
  for row in d.rows():
  	print row

  for column in d.columns():
  	print column
  ......
\end{lstlisting}
The first example could also have been written more compactly as:
\begin{lstlisting}
  for row in d:
  	print row
  ......
\end{lstlisting}

In many cases you do not know which rows in the data file are of interest - in
this case you want to search the data.
\begin{lstlisting}
  d.search('Temperature',4.2)
  d.search('Temperature',4.2,['Temperature','Resistance'])
  d.search('Temperature',lambda x,y: x>10 and x<100)
  d.search('Temperature',lambda x,y: x>10 and
                x<1000 and y[1]<1000,['Temperature','Resistance'])
\end{lstlisting}
The general form is \\\verb:DataFile.search(<search column>,<search term>[,<listof return columns>]):

The first example will return all the rows where the value of the
\textit{Tenperature} column is 4.2. The second example is the same, but only
returns the values from the \textit{Temperature}, and \textit{Resistance}
columns. The rules for selecting the columns are the same as for the
DataFile.column method above -- strings are matched against column headers and
integers select column by number.

The third and fourth examples above demonstrate the use of a function as the
search value. This allows quite complex search criteria to be used. The function
passed to the search routine should take two parameters -- a floating point
number and a numpy array of floating point numbers and should return either
\textit{ture} or \textit{False}. The function is evaluated for each row in the
data file and is passed the value corresponding to the search column as the
first parameter while the second parameter contains a list of all of the values
in the row to be returned. If the search function returns True, then the row is
returned, otherwise it isn't. In thr last example, the final parameter can
either be a list of columns or a single column. The rules for indexing columns
are the same as used for the \textbf{DataFile.find\_col} method.

Sometimes you may want not to get the rows of data that you are looking for as a
separate array, but merely mark them for inclusion (or exclusion) from subsequent
operations. This is where the masked array (see \ref{maskeddata}) comes into its own.
To select which rows of data have been masked off, use the \textbf{filter} method.

\begin{lstlisting}
 d.filter(lambda r:r[0]>5)
 d.filter(lambda r:r[0]>5,['Temp'])
\end{lstlisting}

With jsut a single argument, the filter method takes a complete row at a time and passes it
to the first argument, expecting to get a boolean response (or list olf booleans equal in length
to the number of columns). With a second argument as in the second example, you can sepcify which
columns are passed to the filtering function in what order. The second argument must be a list
of things which can be used to index a column (\ie strings, integers, regular expressions).

\subsubsection{Find out more about the data}

Another question you might want to ask is, what are all the unique
values of data in a given column (or set of columns). The Python numpy
package has a function to do this and we have a direct pass through
from the DataFile object for this:

\begin{lstlisting}
   d.unique('Temp')
   d.unique(column,return_index=False, return_inverse=False)
\end{lstlisting}

The two optional keywords cause the numpy routine to return the
indices of the unique and all non-unique values in the array. The
column is specified in the same way as the \textbf{DataFile.column}
method does.

\subsubsection{Copying Data}

One of the characterisitics of Python that can confuse those used to other
programming languages is that assignments and argument passing is by reference
and not by value. This can lead to unexcted results as you can end up modifying variables you were not expecting ! To help with creating genuine copies of data Python provides the copy module. Whilst this works with DataFile objects, for convenience, the \textbf{DataFile.clone} atribute is provided to make a deep copy of a DataFile object.

\keypoint{This is an attribute not a method, so there are no brackets here !}

\begin{lstlisting}
    t=d.clone
\end{lstlisting}


\subsection{Modifying Data}

\subsubsection{Appending data}

The simplest way to modify some data might be to append some columns or rows.
The Stoner mpodule redefines two standard operators, \verb:+: and \verb:&: to
have special meanings:
\begin{lstlisting}
  a=Stoner.DataFile('some_new_data.txt')
  add_rows=d+a
  add_columns=d&a
\end{lstlisting}
In these example, \textit{a} is a second DataFile object that contains some
data. In the first example, a new DataFile object is created where the contents
of \textit{a} are added as new rows after the data in \textit{d}. Any metadata
that is in \textit{a} and not in \textit{d} are added to the metadata as well.
There is a requirement, however, that the column headers of \textit{d} and
\textit{a} are the same -- \ie that the two DataFile objects appear to represent
similar data.

In the second example, the data in \textit{a} is added as new columns after the
data from \textit{d}. In this case, there is a requirement that the two DataFile
objects have the same number of rows.

These operators are not limited just to DataFile objects, you can also add numpy
arrays to the DataFile object to append additional data.
\begin{lstlisting}
  import numpy as np
  x=np.array([1,2,3])
  new_data=d+x
  y=np.array([1,2,3],[11,12,13],[21,22,23],[31,32,33]])
  new_data=d+y
  z={"X":1.0,"Y":2.1,"Z":7.5}
  new_data=d+z
  new_data=d+[x,y,z]
  column=d.column[0]
  new_data=d&column
\end{lstlisting}
In the first example above, we add a single row of data to \textit{d}. This
assumes that the number of elements in the array matches the number of columns
in the data file. The second example is similar but this time appends a 2
dimensional numpy array to the data. The third example demonstrates adding data from a dictioary. In this case
the keys of the dictionary are used to determine which column the values are added to. If their columns that
don't match one of the dictionary keys, then a \textit{NaN} is inserted. If their are keys that don't match
columns labels, then new columns are added to the data set, filled with \textit{NaN}. In the fourth example, each element
in the list is added in turn to \textit{d}. A similar effect would be achieved by doing \verb#new_data=d+x+y+z#.

The last example appends a numpy array as
a column to \textit{d}. In this case the requirement is that the numpy array has
the same or fewer rows of data as \textit{d}.

\subsubsection{Working with Columns of Data}
\subsubsection{Rearranging Columns of Data}
Sometimes it is useful to rearrange columns of data. \textbf{DataFile} offers a couple of methods to help with this.
\begin{lstlisting}
   d.swap_column(('Resistance','Temperature'))
   d.swap_column(('Resistance','Temperature'),headers_too=False)
   d.swap_column([(0,1),('Temp','Volt'),(2,'Curr')])
   d.reorder([1,3,'Volt','Temp'])
   d.reorder([1,3,'Volt','Temp'],header_too=False)
\end{lstlisting}

The \textbf{swap} method takes either a tuple of column names/indices or a list of such tuples and swaps the columns accordingly, whilst the \textbf{reorder} method takes a list of column labels/indices and constructs a new data matrix out of those columns in the new order. The \textit{headers\_too=False} options, as the name suggests, cause the column headers not be rearranged.

\subsubsection{Renaming Columns of Data}
As a convenience, \textbf{DataFile} also offers a useful method to rename data columns:

\begin{lstlisting}
   d.rename('old_name','new_name')
   d.rename(0,'new_name')
\end{lstlisting}

Alternatively,of course, one could just edit the column\_headers attribute.

\subsubsection{Inserting Columns of Data}
The append columns operator \verb#&# will only add columns to the end of a
dataset. If you want to add a column of data in the middle of the data set then
you should use the \textbf{add\_column} method.

\begin{lstlisting}
  d.add_column(numpy.array(range(100)),'Column Header')
  d.add_column(numpy.array(range(100)),'Column Header',Index)
  d.add_column(lambda x: x[0]-x[1],'Column Header',func_args=None)
\end{lstlisting}

The first example simply adds a column of data to the end of the dataset and
sets the new column headers. The second variant  inserts the new column before
column \textit{Index}. \textit{Index} follows the same rules as for the
\textbf{DataFile.colummn()} method. In the third example, the new column data is
generated by applying the specified function. The function is passed s dingle
row as a 1D numpy array and any of the keyword, argument pairs passed in a
dictionary to the optional \textit{func\_args} argument.

The \textbf{DataFile.add\_column} method returns a copy of the DataFile object
itself as well as modifying the object. This is to allow the metod to be chained
up with other methods for more compact code writing.

\subsubsection{Deleting Rows of Data}

Removing complete rows of data is achieved using the \textbf{DataFile.del\_row}
method.

\begin{lstlisting}
  d.del_rows(10)
  d.del_rows('X Col',value)
  d.del_rows('X Col',lambda x,y:x>300)
\end{lstlisting}

The first variant will delete row 10 from the data set (where the first row will
be row 0). You can also supply a list or slice to \textbf{DataFile.del\_rows} to
delete multiple rows.

If you do not know in advance which row to delete, then the second and third
variants provide more advanced options. The second variant searches for and
deletes all rows in which the specified column contains \textit{value}. The
third variant selects which ros to delete by calling a user supplied function
for each row. The user supplied function is the same in form and definitition as
that used for the \textbf{DataFile.search} method.

\subsubsection{Deleting Columns of Data}

Deleting whole columns of data can be done by referring to a column by index or
column header - the indexing rules are the same as used for the
\textbf{DataFile.column} method.

\begin{lstlisting}
  d.del_column('Temperature')
  d.del_column(1)
\end{lstlisting}

\subsubsection{Sorting Data}

Data can be sorted by one or more columns, specifying the columns as a number or
string for single columns or a list or tuple of strings or numbers for multiple
columns. Currently only ascending sorts are supported.

\begin{lstlisting}
  d.sort('Temp')
  d.sort(['Temp','Gate'])
\end{lstlisting}

\subsection{Saving Data}\label{save}

Only saving data in the \textit{TDI} format and as comma or tab deliminated formats is supported.

\warning{The \textbf{CSVFile} comma or tab deliminated files discard all metadata about the measurement. You absolutely must not use this as your primary data format -- always keep the \textit{TDI} format files as well.}

\begin{lstlisting}
  d.save()
  d.save(filename)
  d=Stoner.CSVFile(d)
  d.save()
  d.save(filename,'\t')
\end{lstlisting}

In the first case, the filename used tosave the data is determined from the
filename attribute of the DataFile object. This will have been set when the
filewas loaded from disc.

If the filename attribute has not been set \eg if the DataFile object was
created from scratch, then the \textbf{DataFile.save} method will cause a dialog
box to be raised so that the user can supply a filename.

In the second variant, the supplied filename is used and the filename attribute
is changed to match this \ie \verb#d.filename# will always return the last
filename used for a load or save operation.

The third is similar but convert the file to \textit{cvs} format while the fourth also specifies that the deliminator is a tab character.

\section{Plotting Data}

Data plotting and visualisation is handled by the PlotFile sub-class of DataFile. The purpose of the methods detailed here is to provide quick and convenient ways to plot data rather than providing publication ready figures.

\begin{lstlisting}
   import Stoner.Plot as plot
   p=plot.PlotFile(d)
\end{lstlisting}

The first line imports the \textbf{Stoner.Plot} module. Strictly, this is unnecessary as the Plot module's namespace is imported when the Stoner package as a whole is imported. The second line creates an instance of the \textbf{PlotFile} class. PlotFile inherits the constructor method of \textbf{DataFile} and so all the variations detailed above work with PlotFile. In particular, the form shown in the second line is a easy way to convert a DataFile instance to a PlotFile instance for plotting.

\subsection{Plotting 2D data}

\textit{x-y} plots are produced by the \textbf{PlotFile.plot\_xy} method:

\begin{lstlisting}
   p.plot_xy(column_x, column_y)
   p.plot_xy(column_x, [y1,y2])
   p.plot_xy(x,y,'ro')
   p.plot_xy(x,[y1,y2],['ro','b-'])
   p.plot_xy(x,y,title='My Plot')
   p.plot_xy(x,y,figure=2)
   p.plot_xy(x,y,plotter=pyplot.semilogy)
\end{lstlisting}

The examples above demonstrate several use cases of the \textbf{plot\_xy} method. The first parameter is always the x column that contains the data, the second is the y-data either as a single column or list of columns. The third parameter is the style of the plot (lines, points, colours \etc) and can either be a list if the y-column data is a list or a single string. Finally additional parameters can be given to specify a title and to control which figure is used for the plot. All matplotlib keyword parameters can be specified as additional keyword arguments and are passed through to the relevant plotting function. The final example illustrates a convenient way to produce log-linear and log-log plots. By default, \textbf{plotxy} uses the \textbf{pyplot.plot} function to produce linear scaler plots. There are a number of useful plotter functions that will work like this:
\begin{description}
  \item[pyplot.semilogx,pyplot.semilogy] These two plotting functions will produce log-linear plots, with semilogx making the x-axes the log one and semilogy the y-axis.
  \item[pyplot.loglog] Liek the semi-log plots, this will produce a log-log plot.
  \item[pyplot.errorbar] this particularly useful plotting function will draw error bars. The values for the error bars are passed as keyword arguments, \textit{xerr} or \textit{yerr}. In standard matplotlib, these can be numpy arrays or constants. \textbf{PlotFile.plot\_xy} extends this by intercepting these arguements and offering some short cuts:
      \begin{lstlisting}
         p.plot_xy(x,y,plotter=errorbar,yerr='dResistance',xerr=[5,'dTemp+'])
      \end{lstlisting}
      This is equivalent to doing something like:
       \begin{lstlisting}
         p.plot_xy(x,y,plotter=errorbar,yerr=p.column('dResistance'),xerr=[p.column(5),p.column('dTemp+')])
      \end{lstlisting}
    If you actually want to pass a constant to the \textit{x/yerr} keywords you should use a float rather than an integer.
\end{description}

The X and Y axis label will be set from the column headers.

\subsection{Plotting 3D Data}

 A number of the measurement rigs will produce data in the form of rows of $x,y,z$ values. Often it is desirable to plot these on a surface plot or 3D plot. The \textbf{PlotFile.plot\_xyz} method can be used for this.

 \begin{lstlisting}
    p.plot_xyz(col_x,col_y,col_z)
    p.plot_xyz(col_x,col_y,col_z,cmap=matplotlib.cm.jet)
    p.plot)xyz(col-x,col-y,col-z,plotter=pyplot.pcolor)
    p.plot_xyz(col_x,col_y,col_z,xlim=(-10,10,100),ylim=(-10,10,100))
 \end{lstlisting}

 By default the plot\_xyz will produce a 3D surface plot with the z-axis coded with a rainbow colourmap (specifically, the matplotlib provided \textit{matplotlib.cm.jet} colourmap. This can be overriden with the \textit{cmap} keyword parameter. If a simple 2D surface plot is required, then the \textit{plotter} parameter should be set to a suitable function such as \textbf{pyplot.pcolor}.

 Like \textbf{plot\_xy}, a \textit{figure} parameter can be used to control the figure being used and any additional keywords are passed through to the plotting function. The axes labels are set from the corresponding column labels.

 Another option is a contour plot based on $(x,y,z)$ data points. This can be done with the \textbf{contour\_xyz} method.

 \begin{lstlisting}
 	p.contour_xyz(xcol,ycol,zcol,shape=(50,50))
 	p.contour_xyz(xcol,ycol,zcol,xlim=(10,10,100),ylim=(-10,10,100))
\end{lstlisting}

Both \textbf{plot\_xyz} and \textbf{contour\_xyz} make use of a call to \textbf{griddata} which is a utility method of the \textbf{PlotFile} -- essentially this is just a pass through method to the underlying \textit{scipy.interpolate.griddata} function. The shape of the grid is determined through a combination of the \textit{xlim}, \textit{ylim} and \textit{shape} arguments.

\begin{lstlisting}
X,Y,Z=p.griddata(xcol,ycol,zcol,shape=(100,100))
X,Y,Z=p.griddata(xcol,ycol,zcol,xlim=(-10,10,100),ylim=(-10,10,100))
\end{lstlisting}

If a \textit{xlim} or \textit{ylim} arguments are provided and are two tuples, then they set the maximum and minimum values of the relevant axis. If they are three tuples, then the third argument is the number of points along that axis and overrides any setting in the \textit{shape} parameter. If the \textit{xlim} or \textit{ylim} parameters are not presents, then the maximum and minimum values of the relevant axis are used. If not shape information is provided, the default is to make the shape a square of sidelength given by the square root of the number of points.

 Alternatively, if your data is already in the form of a matrix, you can use the \textbf{PlotFile.plot\_matrix} method:

 \begin{lstlisting}
    p.plot_matrix()
    p.plot_matrix(xvals,yvals,rectang,title="Title",xlabel="X Axis",ylabel="Y Axis",zlabel="Z Axis",cmap=matplotlib.cm.jet)
    p.plot_matrix(plotter=pyplot.pcolor,figure=False)
 \end{lstlisting}

 The first example just uses all the default values, in which case the matrix is assumed to run from the 2nd column in the file to the last and over all of the rows. The x values for each row are found from the contents of the first column, and the y values for each column are found from the column headers interpreted as a floating pint number. The colourmap defaults to the built in `jet' theme. The x axis label is set to be the column header for the first column, the y axis label is set either from the meta data item ``ylabel'' or to ``Y Data''. Likewise the z axis label is set from the corresponding metadata item or defaults to ``Z Data''. In the second form these parameters are all set explicitly. The \textit{xvals} parameter can be either a column index (integer or sring) or a list, tuple or numpy array. The \textit{yvals} parameter can be either an row number (integer) or list,tuple or numpy array. Other parameters (including \textit{plotter}, \textit{figure} \etc) work as for the \textbf{PlotFile.plot\_xyz} method. The \textit{rectang} parameter is used to select only part of the data array to use as the matrix. It may be 2-tuple in which case it specifies just the origin as (row,column) or a 4-tuple in which case the third and forth elements are the number of rows and columns to include. If \textit{xvals} or \textit{yvals} specify particular column or rows then the origin of the matrix is moved to be one column further over and one row further down (\ie the matrix is to the right and below the columns and rows used to generate the x and y data values). The final example illustrates how to generate a new 2D surface plot in a new window using default matrix setup.

 \subsection{Getting More Control on the Figure}

 It is useful to be able to get access to the matplotlib figure that is used for each \textbf{PlotFle} instance. The \textbf{PlotFile.fig} attribute can do this, thus allowing plots from multiple \textbf{PlotFile} instances to be combined in a single figure.

 \begin{lstlisting}
    p1.plot_xy(0,1,'r-')
    p2.plot_xy(0,1,'bo',figure=p1.fig)
 \end{lstlisting}

 Likewise the \textbf{PlotFile.axes} attribute returns the current axes object of the current figure in use by the \textbf{PlotFile} instance.

 There's a couple of extra methods that just pass through to the pyplot equivalents:

 \begin{lstlisting}
    p.draw()
    p.show()
 \end{lstlisting}



\section{Manipulating and Analysing Data}

Curve fitting, data manipulation, and other analysis functions  is handled by a sub-class of the DataFile object -- AnalyseFile

\begin{lstlisting}
   import Stoner.Analysis as Analysis
   a=Analysis.AnalyseFile('Data')
    a2=Analysis.AnalyseFile()
   a2=d
   a3=Analysis.AnalyseFile(d)
\end{lstlisting}

The first line imports the AnaylseFile class. Since the AnalyseFile is a child
class of DataFile, everything you can do with a DataFile also works with an
AnalyseFile object. The next two lines demonstrate creating a blank AnalyseFile
and then copying all of the data, metadata and column headings from an existing
dataFile object. The final variant shows how to cast one child-class of \textbf{DataFile} into another -- in this case an \textbf{AnalyseFile}.

\subsection{Manipulating Data}

Several methods are provided to assist with common data fitting and preparation tasks, such as normalising columns, adding and subtracting columns.

\begin{lstlisting}
   a.normalise('data','reference',header='Normalised Data',replace=True)
   a.normalise(0,1)
   a.normalise(0,3.141592654)
   a.normalise(0,a2.column(0))
\end{lstlisting}

The \textit{normalise} method simply divides the data column by the reference column. By default the normalise method replaces the data column with the new (normalised) data and appends ``(norm)'' to the column header. The keyword arguments header and replace can override this behaviour. The third variant illustrates normalising to a constant (note, however, that if the second argument is an integer it is treated as a column index and not a constant). The final variant takes a 1D array with the same number of elements as rows and uses that to normalise to. A typical example might be to have some baseline scan that one is normalising to.

\begin{lstlisting}
   a.subtract('A','B'm header="A-B",replace=True)
   a.subtract(0,1)
   a.subtract(0,3.141592654)
   a.subtract(0,a2.column(0))
\end{lstlisting}

As one might expect form the name, the \textit{subtract} method subtracts the second column form the first. Unlike \textit{normalise} the first data column will not be replaced but a new column inserted and a new header (defaulting to column header 1 - column header 2) will be created. This can be overridden with the header and replace keyword arguments. The next two variants of the \textit{subtract} method work in an analogous manner to the \textit{normalise} methods. Finally the \text{add} method allows one to add two columns in a similar fashion:

\begin{lstlisting}
   a.add('A','B',header='A plus B',replace=False)
   a.add(0,1)
   a.add(0,3.141592654)
   a.add(0,a2.column(0))
\end{lstlisting}

For completeness we also have:

\begin{lstlisting}
   a.divide('A','B',header='A/B', replace=True)
   a.multiply('A','B',header='A*B', replace=True)
\end{lstlisting}

with variants that take either a 1D array of data or a constant instead of the B column index.

One might wish to split a single data file into several different data files each with the rows of the original
that have a common unique value in one data column, or for which some function of the complete row determines which datafile
each row belongs in. The \textit{split} method is useful for this case.

\begin{lstlisting}
   a.split('Polarisation')
   a.split('Temperature',lambda x,r:x>100)
   a.split(['Temperature','Polarisation'],[lambda x,r:x>100,None])
\end{lstlisting}

In these examples we assume the \textbf{AnalyseFile} has a data column `Polarisation' that takes two (or more) discrete values
and a column `Temperature' that contains numbers above and below 100.

The first example would return a \textbf{DataFoder} object (see \ref{DataFolder}) containing separate \textbf{AnalyseFile} which
would each contain the rows from the orginal data that had each unique value of the polarisation data. The second example would
produce a \textbf{DataFolder} object containing two \textbf{AnalyseFile} objects for the rows with temperature abobe and below 100.
The final example will result in a \textbf{DataFolder} object that has two groups each of which contains \textbf{AnalyseFile} objects for each
polarisation value.

\subsection{Curve Fitting}

\subsubsection{Simple polynomial Fits}

Simple least squares fitting of polynomial functions is handled by the
\textbf{AnalyseFile.polyfit} method:

\begin{lstlisting}
   a.polyfit(column_x,column_y,polynomial_order, bounds=lambda x, y:True,result="New Column")
\end{lstlisting}

This is a simple pass through to the numpy routine of the same name. The x and y
columns are specified in the first two arguments using the usual index rules for
the Stoner package. The routine will fit multiple columns if \textit{column\_y}
is a list or slice. The polynomial\_order parameter should be a simple integer
greater or equal to 1 to define the degree of polynomial to fit. The bounds
function follows the same rules as the bounds function in
\textbf{DataFile.search} to restrict the fitting to a limited range of rows. The
method returns a list of co-efficients with the highest power first. If
\textit{column\_y} was a list, then a 2D array of co-efficients is returned.

If \textit{result} is specified then a new column with the header given by the result parameter will be created and the fitted polynomial evaluated at each point.

\subsubsection{Simple function fitting}

For more general curve fitting operations the \textbf{AnalyseFile.cruve\_fit}
method can be employed. Again, this is a pass through to the numpy routine of
the same name.

\begin{lstlisting}
   a.curve_fit(func,  xcol, ycol, p0=None, sigma=None,bounds=lambda x, y: True, result=True,replace=False,header="New Column" )
\end{lstlisting}

The first parameter is the fitting function. This should have prototype
\\\verb:y=func(x,p[0],p[1],p[2]...): where p is a list of fitting parameters.
The \textit{p0} parameter contains the initial guesses at the fitting
parameters, the default value is 1. \textit{xcol} and \textit{ycol} are the x
and y columns to fit. This method cannot handle multiple y columns.
\textit{sigma}, if present, provides the weightings for each datapoint and so
should also be an array of the same length as the x and y data. Fianlly, the
bounds function can be used to restrict the fitting to only a subset of the rows
of data.

\textbf{AnalyseFile.curve\_fit} returns a list of two arrays \verb:[popt,pcov]:
where \textit{popt} is an array of the optimal fitting parameters and
\textit{pcov} is a 2D array of the co-variances between the parameters.

If \textit{result} is not \textbf{None} then the fitted data is added to the \textbf{AnalyseFile}
object. Where it is added depends on the combination of the \textit{result}, \textit{replace}
and \textit{header} parameters. If \textit{result} is a string or integer it is interpreted as a column
index at which the fitted data will be inserted (\textit{replace} False) or overwritten over the existing data (\textit{replace} False).
The fitted data will be given the column header \textit{header} unless \textit{header} is not a string, in which ase the column
will be called `Fitted with ' and the name of the function \textit{func}.


\subsubsection{Fitting with limits}

For cases where one requires more flexibility in fitting data, in particular
where the fitting parameters are constrained, the \textbf{AnalyzeFile.mpfit}
method is provided. This is a pass through to the \textbf{mpfit} module.

\begin{lstlisting}
   a.mpfit(func,  xcol, ycol, p_info,  func_args=dict(), sigma=None,bounds=lambda x, y: True, **mpfit_kargs )
\end{lstlisting}

In this case, the \textit{func} argument takes a slightly different
prototype:\\\verb:def func(x,parameters, **func_args): where \textit{parameters}
is a list of the fitting parameters and \textit{func\_args} provides a
dictionary of fixed \ie non-fitting parameters. \textit{xcol} and \textit{ycol}
are the column indices for the x and y data, \textit{bounds} is a bounding
function to select only those rows to use for fitting the function, and
\textit{sigma} are the weightings for each datapoint. The remaining arguments
are a dictionary of keywords to pass through to the \textbf{mpfit} routine and
\textit{p\_info} which is a list of dictionaries which is used to control the
parameters in the fit. This described below.

\textit{p\_info} contains one element for each parameter used to fit the data.
Each element is a dictionary with the following keys:
\begin{description}
  \item[value] the starting parameter value (but see the START\_PARAMS parameter
for more information).
  \item[fixed] a boolean value, whether the parameter is to be held fixed or
not.  Fixed parameters are not varied by MPFIT, but are passed on to MYFUNCT for
evaluation.
  \item[limited] a two-element boolean array.  If the first/second element is
set, then the parameter is bounded on the lower/upper side.  A parameter can be
bounded on both sides.  Both LIMITED and LIMITS must be given together.
  \item[limits] a two-element float array.  Gives the parameter limits on the
lower and upper sides, respectively.  Zero, one or two of these values can be
set, depending on the values of LIMITED.  Both LIMITED and LIMITS must be given
together.
  \item[parname] a string, giving the name of the parameter.  The fitting code
of MPFIT does not use this tag in any way.  However, the default iterfunct will
print the parameter name if available.
  \item[step] the step size to be used in calculating the numerical derivatives.
 If set to zero, then the step size is	computed automatically.  Ignored when
AUTODERIVATIVE=0.
  \item[mpside] the sidedness of the finite difference when computing numerical
derivatives.  This field can take four values:\\
    \begin{description}
      \item [0]one-sided derivative computed automatically
      \item [1]one-sided derivative $(f(x+h) - f(x)  )/h$
      \item [-1] one-sided derivative $(f(x)   - f(x-h))/h$
      \item [2] two-sided derivative $(f(x+h) - f(x-h))/(2*h)$
    \end{description}
	Where H is the STEP parameter described above.  The "automatic"
one-sided derivative method will chose a direction for the finite difference
which does not 			 violate any constraints.  The other methods do
not perform this check.  The two-sided method is in principle more precise, but
requires twice as many function evaluations.  Default: 0.
  \item[mpmaxstep] the maximum change to be made in the parameter value.  During
the fitting process, the parameter will never be changed by more than this value
in one iteration.\\ A value of 0 indicates no maximum.  Default: 0.
  \item[tied] a string expression which ``ties'' the parameter to other	free or
fixed parameters.  Any expression involving	constants and the parameter
array P are permitted.Example: if parameter 2 is always to be twice parameter 1
then use the following: \verb#parinfo(2).tied = '2 * p(1)'#. Since they are
totally constrained, tied parameters are considered to be fixed; no errors are
computed for them.[ NOTE: the PARNAME can't be used in expressions. ]
  \item[mpprint] if set to 1, then the default iterfunct will print the
parameter value.  If set to 0, the parameter value will not be printed.  This
tag can be used to selectively print only a few parameter values out of
many.\\Default: 1 (all parameters printed)
\end{description}

\subsection{More AnalyseFile Functions}

\subsubsection{Applying an arbitary function through the data}

\begin{lstlisting}
 a.apply(func, col, replace = True, header = None)
\end{lstlisting}

Here \textit{func} is an arbitrary function that will take a complete row in the form of a numpy 1D array, \textit{col}
is the index of a column at which the resulting data is to be inserted or overwrite the existing data (depending on the values of \textit{replace} and \textit{header}).


\subsubsection{Basic Data Inspection}
\begin{lstlisting}
   a.max(column)
   a.min(column)
   a.max(column,bounds=labda x,y:y[2]>1 and y[2]<10)
   a.min(column,bounds=labda x,y:y[2]>1 and y[2]<10)
\end{lstlisting}

Hopefully all of the above are fairly obvious ! In the last two cases, one can use a function to limit the search to particular rows (\eg to search for the maximum y value subject to some constraint in x). One important point to note is that the routines return a tuple of two numbers, the maximum (or minimum) and the row number where the maximum or minimum was found.

There are a couple of related functions to help here:

\begin{lstlisting}
   a.span(column)
   a.span(column, bounds=lambda x,y:y[2]>100)
   a.clip(column,(max_v,min_v)
   a.clip(column,b.span(column))
\end{lstlisting}

The span method simply returns a tuple of minimum and maximum values within either the whole column or bounded data. Internally this is just calling the \textbf{max} and \textbf{min} methods. The \textbf{clip} method deletes rows for which the specified column as a value that is either larger or smaller than the maximum or minimum value within the second argument. This allows one to specify either a tuple -- \eg the result of the \textbf{span} method, or a complete list as in the last example above. Specifying a single float would have the effect of removing all rows where the column didn't equal the float value. This is probably not a good idea...

It is worth pointing out that these functions will respect the existing mask on the data unless the bounds parameter is set, in which case the mask is temproarily discarded in favour of one generated from the bounds expression. This can be worked around, however, as the parameter passed to the bounds function is itself a masked array and thus one can include a test of the mask in the bounds function:

\begin{lstlisting}
   a.span(column,bounds=lambda x,y:y[2]>10 or not numpy.any(y.mask))
\end{lstlisting}

\subsubsection{Thresholding and Interpolating Data}
\begin{lstlisting}
   a.threshold(col, threshold, rising=True, falling=False,all_vals=False)
\end{lstlisting}

\begin{lstlisting}
   a.interpolate(newX,kind='linear' )
\end{lstlisting}

\subsubsection{Smoothing and Differentiating Data}

\subsubsection{Peak Finding}

\subsection{Non-linear curve fitting with initialisation file}

If you wish to fit your data to a non-linear function more complicated than a polynomial you can use \verb#Stoner.nlfit.nlfit(inifile, func, data=None)# or equivalently if you have an AnalyseFile instance of your data called d say you can call \verb#d.nlfit(inifile, func)#. This performs a non-linear least squares fitting algorithm to your data and returns the AnalyseFile instance used with an additional final column that is the fit, it also plots the fit. There is an example run script, ini file and data file in PythonCode\textbackslash Scripts, have a look at them to see how to use this function.

The function to fit to can either be created by the user and passed in or one of a library of current existing functions can be used from the FittingFunctions.py file in Stoner\textbackslash src (just pass in the name of the function you wish to use as a string). The function takes it's fitting parameters information from a .ini file created by the user, look at the example .ini file mentioned above for the format, you can see that it allows for the parameters to be fixed or constrained which can be very useful for fitting.

Current functions existing in FittingFunctions.py:
\begin{itemize}
\item Various tunnelling I-V models including BDR, Simmons, Field emission and Tersoff Hamman STM.
\item 2D weak localisation
\item Strijkers model for PCAR fitting
\end{itemize}
Please see the function documentation in FittingFunctions.py for more information about these models. Please do add functions you think would be of use to everybody, have a look at the current functions for examples, the main thing is that the function must take an x array and a list of parameters, apply a function and then return the resulting array.

\section{Working with Lots of Files}\label{DataFolder}

A common case is that you have measured lots of data curves and now have a large stack of data files sitting in a tree of folders on disc and now need to process all of them with some code. The \textbf{DataFolder} class is designed to make it easier to process lots of files.

\subsection{Getting a List of Files}

The first thing you probably want to do is to get a list of data files in a directory (possibly including its subdirectories) and probably matching some sort of filename pattern.

\begin{lstlisting}
   from Stoner.folders import DataFolder
   f=DataFolder(pattern='*.dat')
\end{lstlisting}

In this very simple example, the \textbf{DataFolder} class is imported in the first line and then a new instance \textit{f} is created. The optional \textit{pattern} keyword is used to only collect the files with a .dat extension. In this example, it is assumed that the files are readable by \textbf{DataFile}, if they are in some other format then the \textit{type} keyword can be used:

\begin{lstlisting}
   from Stoner.FileFormats import XRDFile
   f=DataFolder(type=XRDFile,pattern='*.dql')
\end{lstlisting}

To specify a particular directory to look in, simply give the directory as the first argument - otherwise the current duirectory will be used.

\begin{lstlisting}
   f=DataFolder('/home/phygbu/Data',pattern='*.tdi')
\end{lstlisting}

By default the \textbf{DataFolder} constructor will perform a recursive drectory listing of the working folder. Each sub-directory is given a separate \textit{group} within the structure. This allows the \textbf{DataFolder} to logically represent the on-disc layout of the files. The resulting list of files can be accessed via the \textit{files} attribute and sub groups with the \textit{group} attribute (see \ref{groups}:

\begin{lstlisting}
   f.files
\end{lstlisting}

\warning{In some circumstances entries in the \textit{f.files} attribute can be \textbf{DataFile} objects rather than strings. If you want to ensure that you get a list of strings representing the filenames, use \textit{f.ls} instead.}

If you don't want the file listing to be recursive, this can be suppressed by using the \textit{recursive} keyword argument and the file listing can be suppressed altogether with the \textit{nolist} keyword:

\begin{lstlisting}
   f=DataFolder(pattern='*.dat',recursive=False)
   f2=DataFolder(nolist=True)
\end{lstlisting}

If you don't want to create groups for each sub-directory, then set the keyword parameter \textit{flatten} True, or call the \textit{flatten()} method. You can also use the \textit{prune} method to remove groups (including nested groups) that have no data files in them.

\begin{lstlisting}
	f.prune()
	f.flatten()
\end{lstlisting}

The current root directory and pattern are stored in the \textit{directory} and \textit{pattern} keywords and the \textbf{getlist} method can be used to force a new listing of files.

\begin{lstlisting}
   f.dirctory='/home/phygbu/Data'
   f.pattern='*.txt'
   f.getlist()
\end{lstlisting}

Sometimes a more complex filename matching mechanism than simple ``globbing'' is useful. The \textit{patter} keyword can also be a compiled regular expression:

\begin{lstlisting}
   import re
   p=re.compile('i10-\d*.dat')
   f=DataFolder(pattern=p)
   p2=re.compile('i10-(?P<run>\d*)')
   f=DataFolder(pattern=p)
   f[0]['run']
\end{lstlisting}

The second case illustrates a useful feature of regular expressions - they can be used to capture parts of the matched pattern -- and in the python version, one can name the capturing groups. In both cases above the \textbf{DataFolder} has the same file members (basically these would be runs produced by the i10 beamline at Diamond), but in the second case the run number (which comes after ``i10-'' would be captured and presented as the ``run'' parameter in the metadata when the file was read. \warning{Note that the files are not modified - the extra metadata is only added as the file is read by the \textbf{DataFlder}}. The loading process will also add the metadata key ``Loaded From'' to the file which will give you a note of the filename used to read the data.

Finally, akin to \textbf{DataFile} you can force a dialog box to select a directory by passing \textit{False} into the constructor or getlist methods in place of a directory name.

\subsection{Doing Something With Each File}

A \textbf{DataFolder} is an object that you can iterate over, lading the \textbf{DataFile} type object for each of the files in turn. This probides an easy way to run through a set of files, performing the same operation on each:

\begin{lstlisting}
folder=DataFolder(pattern='*.tdi')
for f in folder:
	f=AnalyseFile(f)
  	f.normalise('mac116','mac119')
 	f.save()
\end{lstlisting}

or even more compacts:

\begin{lstlisting}
  [f.normalise('mac116','macc119').save() for f in DataFolder(pattern='*.tdi',type=AnalyseFile)]
\end{lstlisting}

\textbf{DataFolder} is also indexable and has a length:

\begin{lstlisting}
   f=DataFolder()
   len(f)
   f[0]
   f['filename']
\end{lstlisting}

For the second case of indexing, the cose will search the list of filenames for a matching file and return that (roughly equivalent to doing \verb#f[f.files.index("filename")]#)

If you want to know the filenames of all the files in the DataFolder then there is a handy attributes:
\begin{lstlisting}
 f.ls
 f.basenames
\end{lstlisting}

The difference between these two is that \textit{f.basenames} will return only the file part of the filename whilst \textit{f.ls} returns the complete path from the root directory.

\subsection{Sorting, Filtering and Grouping Data Files}\label{groups}

The order of the files in a \textbf{DataFolder} is arbitary. If it is important to process them in a given order then the \textit{sort} method can be used:

\begin{lstlisting}
   f.sort()
   f.sort('tmperature')
   f.sort('Temperature',reverse=True)
   f.sort(lambda x:len(x))
\end{lstlisting}

The first variant simply sorts the files by filename. The second and third variants both look at the ``temperature'' metadata in each file and use that as the sort key. In the third variant, the \textit{revers} keyword is used to reverse the order of the sort. In the final variant, each file is loaded in turn and the supplied function is called and evaluated to find a sort key.

The \textbf{filter} method can be used to prune the list of files to be used by the \textbf{DataFoler}:

\begin{lstlisting}
   f.filter('[ab]*.dat')
   import re
   f.filter(re.compile('i10-\d*\.dat'))
   f.filter(lambda x: x['Temperature']>150)
   f.filter(lambda x: x['Temperature']>150,invert=True)
\end{lstlisting}

The first form performs the filter on the filenames (using the standard python fnmatch module). One can also use a regular expression as illustrated int he second example -- although unlike using the \textit{pattern} keyword in \textbf{getlist}, there is no option to capture metadata (although one could then subsequently set the pattern to achieve this). The third variant calls the supplied function, passing the current file as a \textbf{DataFile} object in each time. If the function evaluates to be True then the file is kept. The \textit{invert} keyword is used to invert the sense of the filter (a particularly silly example here, since the greater than sign could simply be replaced with a less than or equals sign !).

One of the more common tasks is to group a long list of data files into separate groups according to some logical test --  for example gathering files with magnetic field sweeps in a positive direction together and those with magnetic field in a negative direction together. The \textbf{group} method provides a powerful way to do this. Suppose we have a series of data curves taken at a variety of temperatures and with three different magnetic fields:

\begin{lstlisting}
   f.group{'temperature'}
   f.group(lambda x:"positive" if x['B-Field']>0 else "negative")
   f.group(['temperature',lambda x:"positive" if x['B-Field']>0 else "negative"])
   f.groups
\end{lstlisting}

The \textbf{group} method splits the files in the \textbf{DataFolder} into several groups each of which share a common value of the arguement supplied to the \textbf{group} method. A group is itself another isntanceinstance of the \textbf{DataFolder} class. Each \textbf{DataFolder} object maintains a dictionary called \textit{groups} whose keys are the distinct values of the argument of the \textbf{group} methods and whose values are \textbf{DataFolder} objects. So, if our \textbf{DataFolder} f contained files measured at 4.2, 77 and 300K and at fields of 1T and -1T then the first variant would create 3 groups: 4.2, 77 and 300 each one of which would be a \textbf{DataFolder} object containg the files measured at those temperatures. The second varaint would produce 2 groups -- ``postive'' containing the files measured with magnetic field of 1T and ``negative'' containing the files measured at -1T. The third variant then goes one stage further and would produce 3 groups, each of which in turn had 2 groups. The groups are accessed via the \textit{group} attribute:

\begin{lstlisting}
   f.groups[4.2].groups["positive"].files
\end{lstlisting}

would return a list of the files measured at 4.2K and 1T.

If you try indexing a \textbf{DataFolder} with a string and there is no file with as it's filename and there is a group with a key of the same string then \textbf{dataFolder} will return the correspondign group. This allows a more compact navigation through an xtended group structure.

\begin{lstlisting}
 f.group(['project','sample','device'])
 f['ASF']['ASF038']['A']
\end{lstlisting}

If you just ant to create a new empty group in your \textbf{DataFoler}, you can use the \textit{add\_group} method.
\begin{lstlisting}
f.add_group("key_value")
\end{lstlisting}
which will create the new group with a key of ``key\_value''.

One task you might want to do would be to work through all the groups in a \textbf{DataFolder} and run some function either with each file in the group or on the whole group. This is further complicated if you want to iterate over all the sub-groups within a group. The \textit{walk\_groups()} method is useful here.
\begin{lstlisting}
 f.walk_groups(func,group=True,replace_terminal=True,walker_args={"arg1":"value1"})
\end{lstlisting}

This will iterate over the complete hierarchy of groups and sub groups in the folder and execute the function \textit{func} once for each group. If the \textit{group} parameter is False then it will execute \textit{func} once for each file. The function \textit{fun} should be defined something like:
\begin{lstlisting}
 def func(group,list_of-group_keys,arg1,arg2...)
\end{lstlisting}
The first parameter should expect and instance of \textbf{PDataFile} if \textit{group} is False or an instance of \textbf{DataFolder} if \textit{group} is True. The second parameter will be given a list of of strings representing the group key values from the topmost group to the lowest (terminal) group.

The \textit{replace\_terminal} parameter applies when \textit{group} is True and the function returns a \textbf{DataFile} object. This indicates that the group on which the function was called should be removed from the list fo groups and the returned \textbf{DataFile} object should be added to the list of files in the folder. This operation is useful when one is processing a group of files to combined them into a single dataset. Combining a multi-level grouping operation and successive calls to \textit{walk\_groups} can rapidly reduce a large set of data files representing a multi-dimensional data set into a single file with minimal coding.


In some cases you will want to work with sets of files coming from different groups in order. For example, if above we had a sequence of 10 data files for each field and temperature and we wanted to process the positive and negative field curves together for a given temperature in turn. In this case the \textbf{zip\_groups} method can be useful.

\begin{lstlisting}
   f.groups[4.2].zip_groups(['positive','negative'])
\end{lstlisting}

This would return a list of tuples of DataFile objects where the tuples would be the first positive and first negative field files, then the second of each, then third of each and so. This presupposes that the files started of sorted by some suitable parameter (\eg a gate voltage).



\section{Cookbook}

This section gives some short examples to give an idea of things that can be
done with the Stoner python module in just a few lines.

\subsection{The Utils module}

The \textbf{Stoner} package comes with an extra \textbf{Utils} module that includes some handy utility functions.
So far the module just contains one function that will take a single \textbf{dataFile} object and split it
into a series of \textbf{DataFile} objects where one column is either rising or falling. This is designed to help
deal with analysis problems involving hysteretic data.
\begin{lstlisting}
from Stoner.Utils import split_up_down
folder=split_up_down(data,column)
\end{lstlisting}
\textit{folder} is a DataFolder instance with two groups, one for rising values of the column and one for falling values of the column. The \textit{split\_up\_down} will take an optional third parameter which is an existing \textbf{DataFolder} instance to which the new groups (if they don't already exist) and files will be added.

\subsection{Extract X-Y(Z) from X-Y-Z data}

In a number of measurement systems the data is returned as 3 parameters X, Y and
Z and one wishes to extract X-Y as a function of constant Z. For example, $I-V$
sweeps as a function of gate voltage $V_G$. Assuming we have a data file with
columns \textit{Current}, \textit{Voltage},\textit{Gate}:

\begin{lstlisting}
   d=DataFile('data.txt')
   t=d
   for gate in d.unique('Gate'):
       t.data=d.search('Gate',gate)
       t.save('Data Gate='+str(gate)+'.txt')
\end{lstlisting}

The first line opens the data file containing the $I-V(V_G)$ data. The second
creates a temporary copy of the DataFile object - ensuring that we get a copy of
all metadata and column headers. The \textbf{for} loop iterates over all unique
values of the data in the gate column and then inside the for loop, searches for
the corresponding $I-V$ data, sets it as the data of the temporary DataFile and
then saves it.

\subsection{Mapping X-Y-Z data to Z(X,Y) data}

In a similar fashion to the previous section, where data has been recorded with
fixed values of $X$ and $Y$ \eg $I$ measured for fixed $V$ and $V_G$, it can be
useful to map the data to a matrix.

\begin{lstlisting}
   d=DataFile('Data,.txt')
   t=d
   for gate in d.unique('Gate'):
      t=t+d.search('Gate',gate)[:,d.find_col('Current')]
   t.column_headers=['Bias='+str(x) for x in d.unique('Voltage')]
   t.add_column(d.unique('Gate'),'Gate Voltage',0)
\end{lstlisting}

The start of the script follows the previous section, however this time in the
for loop the addition operator is used to add a single row to the temporary
DataFile \textit{t}. In this case we are using the utility method
\textbf{DataFile.find\_col} to find the index of the column with the current
data. After the for loop we set the column headers in \textit{t} and then insert
an additional column at the start with the gate voltage values.

The matrix generated by this code is suitable for feeding directly into \textbf{PlotFile.\linebreak plot\_matrix()}, however, the same plot could be generated directly from the \linebreak\textbf{PlotFile.plot\_xyz()} method too.

\section{Developer's Guide}

This section provides some notes and guidance on extending the Stoner Package.

\subsection{Adding New Data File Types}

The first question to ask is whether the data file format that you are working with is one that others in the group will be interested in using. If so, then the best thing would be to include it in the \textbf{fileFormats} module in the package, otherwise you should just write the class in your own script files. In either case, develop the class in your own script files first.

The best way to implement handling a new data format is to write a new subclass of DataFile:
 \begin{lstlisting}
class NewInstrumentFile(DataFile):
    """Extends DataFile to load files from somewhere else

    Written by Gavin Burnell 11/3/2012"""
 \end{lstlisting}
 A document string should be provided that will help the user identify the function of the new class (and avoid using names that might be commonly replicated !). Only one method needs to be implemented: a new \textit{load}method. The \textit{load} method should have the following structure:
\begin{lstlisting}
      def load(self,filename=None,*args):
        """Just call the parent class but with the right parameters set"""
        if filename is None or not filename:
            self.get_filename('r')
        else:
            self.filename = filename
\end{lstlisting}
then follows the code to actually read the file. It \underline{must} at the very least provide a column header for every column of data and read in as much numeric data as possible and it \underline{should} read as much of the meta data as possible. The function terminates by returning a copy of the current object:
\begin{lstlisting}
        return self
\end{lstlisting}
One useful function for reading metadata from files is \textit{self.metadata.string\_to\_type()} which will try to convert a string representation of data into a sensible Python type.

There is one global attribute that can be used to tweak the automatic file importing code.
\begin{lstlisting}
      f.priority=32
\end{lstlisting}

When the subclasses are tried to see if they can load an undetermined file, they are tried in order of priority. If your load code can make a positive determination that it has the correct file (\eg by looking for some magic combination of characters at the start of the file) and can throw an exception if it tries loading an incorrect file, then you can give it a lower priority number to force it to run earlier. Conversely if your only way of identifying your own files is seeing they make sense when you try to load them and that you might partially succeed with a file from another system (as can happen if you have a tab or comma separated text file), then you should raise the priority number. Currently \textbf{DataFolder} defaults to 32, \textbf{CSVFile} and \textbf{BigBlueFile} have values of 128 and 64 respectively.

If you need to write any additional methods please make sure that they have DoxyGen document strings so that the API documentation is picked up correctly.



\end{document}
